\input{../math104.tex}
\usepackage{amsmath, dsfont}

\oddsidemargin 0in
\evensidemargin 0in
\textwidth 6.5in
\topmargin -0.5in
\textheight 9.0in
\newcommand{\norm}[1]{\left\lVert #1 \right\rVert}
\newcommand{\?}{\stackrel{?}{=}}

\begin{document}

\solution{Nikhil Unni}{Assignment \#1}{Spring 2016}
\pagestyle{myheadings}

\begin{itemize}
  \item [1.1]
    Prove $1^2 + 2^2 + \cdots + n^2 = \frac{1}{6} n (n+1)(2n+1)$ for all positive integers n.\\\\

    We can prove this with the principle of mathematical induction.
    
    Let $P_n$ denote whether or not the statement is true for some positive integer n. 

    Then $P_1 \equiv 1^2 = \frac{1}{6} * 1(1 + 1)(2*1 + 1) = \frac{6}{6} = 1$.

    So next we have to show that for any $P_{n+1}$, if $P_n$ is true, that $P_{n+1}$ must be true as well. So we can make the following substitution:\\
    $$(1^2 + 2^2 + \cdots + n^2) + (n+1)^2 = \frac{1}{6}n(n+1)(2n+1) + (n+1)^2$$
    Then:
    $$\frac{n(n+1)(2n+1) + 6(n+1)^2}{6} = \frac{(n+1)[(n(2n+1) + 6(n+1))]}{6}$$
    $$ = \frac{(n+1)(2n^2+7n+6)}{6}$$
    $$ = \frac{(n+1)(n+2)(2n+3)}{6} = \frac{1}{6}(n+1)(n+2)(2(n+1) + 1)$$
    Since we've proved the statement for $P_1$, and that any $P_{n+1}$ is true when $P_n$ is true, by the principle of mathematical induction, the statement is true for all positive integers n.\\\\


  \item [1.8]
    The principle of mathematical induction can be extended as follows. A list $P_m, P_{m+1}, \cdots$ of prepositiions is true provided (i) $P_m$ is true, (ii) $P_{n+1}$ is true whenever $P_n$ is true and $n \geq m$.
    \item [a] Prove $n^2 > n + 1$ for all integers $n \geq 2$. \\\\
      
      Let $P_k$ be the statement that $k^2 > k + 1$ for some $k \geq 2$.
      \begin{itemize}
        \item
          $2^2 > 2 + 1 \equiv 4 > 3$\\
          So the statement $P_2$ is correct.
        \item
          If $P_n$ is true, we can show that $P_{n+1}$ is true, for some $n \geq 2$.
          $$(n+1)^2 = n^2 + 2n + 1 > n + 1 + 2n + 1 = 3n + 2 > n + 2$$
          $2 = 2$ and $3n$ must be greater than $n$ for a positive number $n$ (which it is because $n \geq 2$).\\          
      \end{itemize}

      By the extension of induction, the statement is true for all $n \geq 2$.\\\\

    \item [b] Prove $n! > n^2$ for all integers $n \geq 4$.\\\\
      \begin{itemize}
        \item 
          $4! = 24 > 4^2 = 16$. So the statement $P_4$ is correct.

        \item
          If $P_n$ is true, we can show that $P_{n+1}$ is true, for some $n \geq$. So we know that:
          $$(n+1)! = n!(n+1) > n^2(n+1)$$
          From part (a) we know that $n^2 > n + 1$, and since $n \geq 4 \geq 2$, we know this is true for any of our $n$.
          $$(n+1)! > n^2(n+1) > (n+1)(n+1) = (n+1)^2$$ \\\\
      \end{itemize}
  \item [1.11]
    For each $n \in \mathds{N}$, let $P_n$ denote the assertion ``$n^2 + 5n + 1$ is an even integer.''
    \item [a] Prove $P_{n+1}$ is true whenever $P_n$ is true.\\\\
      If $P_n$ is true, then $n^2 + 5n + 1$ is even and can be represented as $2k$, for some $k \in \mathds{Z}$.
      Then:
      $$(n+1)^2 + 5(n+1) + 1 = n^2 + 2n + 1 + 5n + 5 + 1 = (n^2+5n+1) + 2n + 6 = 2k + 2n + 6 = 2(k + n + 6)$$
      Since $k + n + 6$ is an integer (since addition of integers is closed), $(n+1)^2 + 5(n+1) + 1$ is even, and so $P_{n+1}$ is true if $P_n$ is true.\\

    \item [b] For which n is $P_n$ actually true? What is the moral of this exercise?\\\\

      It's not true for any integer. (For any odd integer, you have the addition of 3 odd numbers, and for any even number you have the addition of 2 even numbers and an odd number, always yielding an odd sum). The moral of the exercise is that you cannot inductively prove anything without a base case, even if you can prove the recursive case.
  \item [1.12]
    \begin{itemize}
    \item [a] 
      Verify the binomial theorem for $n = $ 1, 2, and 3.\\\\
      
      $$(a+b)^1 = \binom{1}{0}a^1 + \binom{1}{1}b^1 = \frac{1!}{0!1!}a + \frac{1!}{1!0!} = a + b$$
      $$(a+b)^2 = \binom{2}{0}a^2 + \binom{2}{1}ab + \binom{2}{2}b^2 = \frac{2!}{0!2!}a^2 + \frac{2!}{1!1!}ab + \frac{2!}{2!0!}b^2 = a^2 + 2ab + b^2$$
      $$(a+b)^3 = \binom{3}{0}a^3 + \binom{3}{1}a^2b + \binom{3}{2}ab^2 + \binom{3}{3}b^3 = \frac{3!}{0!3!}a^3 + \frac{3!}{1!2!}a^2b + \frac{3!}{2!1!}ab^2 + \frac{3!}{3!0!}b^3 = a^3 + 3a^2b + 3ab^2 + b^3$$
    \item [b]
      Show $\binom{n}{k} + \binom{n}{k-1} = \binom{n+1}{k}$ for $k = 1,2,\cdots, n$.\\\\

      $$\binom{n}{k} + \binom{n}{k-1} = \frac{n!}{k!(n-k)!} + \frac{n!}{(k-1)!(n-k+1)!} = \frac{n!}{k(k-1)!(n-k)!} + \frac{n!}{(k-1)!(n-k+1)(n-k)!}$$
      Grouping together with a common denominator we get:
      $$\frac{n!(n-k+1)}{k(k-1)!(n-k)!(n-k+1)} + \frac{n!k}{k(k-1)!(n-k+1)(n-k)!}$$
      $$=\frac{n!(n-k+1+k)}{k!(n-k+1)!} = \frac{(n+1)!}{k!(n+1-k)!}$$
      $$=\binom{n+1}{k} \qed$$

    \item [c]
      Prove the binomial theorem using mathematical induction and part (b).\\\\

      Let $P_k$ be the assertion that the binomial theorem is valid for some $k \in \mathds{N}$.\\
      From part (a), we've already proven $P_1, P_2,$ and $P_3$. So we have to show that $P_{n+1}$ is true given that $P_n$ is true.\\
      
      In summation notation, $P_n$ states that $(x+y)^n = \Sigma_{k=0}^n \binom{n}{k} x^ky^{n-k}$.\\
      So then:
      $$(x+y)^{n+1} = (x+y)(x+y)^n = (x+y)(\Sigma_{k=0}^n \binom{n}{k} x^ky^{n-k})$$
      Distributing the x and y we get:
      $$= \Sigma_{k=0}^n \binom{n}{k} x^{k+1}y^{n-k} + \Sigma_{k=0}^n \binom{n}{k} x^ky^{n-k+1}$$
      $$= \Sigma_{k=0}^n \binom{n}{k} x^{k+1}y^{n-k} + \Sigma_{k=0}^n \binom{n}{k} x^ky^{n-k+1}$$
    \end{itemize}
    
\end{itemize}

\end{document}
