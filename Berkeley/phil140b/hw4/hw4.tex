\input{../phil140b.tex}
\usepackage{amsmath, dsfont, mathtools, verbatim, tikz, float, textcomp, mathrsfs}

\usetikzlibrary{arrows,automata}

\oddsidemargin 0in
\evensidemargin 0in
\textwidth 6.5in
\topmargin -0.5in
\textheight 9.0in
\newcommand{\norm}[1]{\left\lVert #1 \right\rVert}
\newcommand{\abs}[1]{\left\vert #1 \right\vert}
\newcommand{\?}{\stackrel{?}{=}}
\DeclarePairedDelimiter{\ceil}{\lceil}{\rceil}




\begin{document}

\solution{Nikhil Unni}{Assignment \#3}{Spring 2016}
\pagestyle{myheadings}

\begin{enumerate}
  \item 
    \begin{question}
      Show that the function $f(x)$ defined by:
      \[ \begin{cases} 
        x^2 & x \text{  is even}\\
        x+1 & x \text{  is odd}\\
      \end{cases}
      \]
      is primitive recursive.
    \end{question}

    First, let's define the (characteristic function of a) relation $R(x)$ to see if a number is odd. We can define it as:
    $$R(x') = 1 - R(x)$$
    $$R(z(x)) = 0$$

    So $R(x)$ is primitive recursive.

    Then, we can simply define an $f(x)$ that is primitive recursive:
    $$f(x) = R(x)(x+1) + (1 - R(x))(x*x)$$
    Since $f$ was constructed with only primitive recursive functions, it is also primitive recursive.
  \item
    \begin{question}
      Let $f(x_1,\cdots,x_n,y)$ be a function. Define $\sum_{y < z} f(x_1,\cdots,x_n,y)$ to be $f(x_1,\cdots,x_n,0) + \cdots + f(x_1,\cdots,x_n,z-1)$ if $z \neq 0$ and $0$ if $z=0$. Morover, define $\Pi_{y<z} f(x_1,\cdots,x_n,y)$ to be equal to $f(x_1,\cdots,x_n,0) \cdots f(x_1,\cdots,x_n,z-1)$ if $z \neq 0$ and equals $1$ if $z=0$.\\
      The class of elementary functions is the smallest class which contains $x+y$, $xy$, $\abs{x-y}$, $id^{n}_{i}(x_1,\cdots,x_n)$, $x/y$, and is closed under composition, bounded sums, and bounded products.\\
      Show that the following functions are elementary:
    \end{question}

    \begin{enumerate}
      \item [(i)]
        \begin{question} $z(x)$ \end{question}
        $$z(x) = \abs{x-x}$$
      \item [(ii)]
        \begin{question} $s(x)$ \end{question}
        $$one(x) = \Pi_{y < 0} z(x)$$
        $$s(x) = x + one(x)$$
      \item [(iii)]
        \begin{question} $sg(x)$ \end{question}
        $$sg(0) = 0$$
        $$sg(s(y)) = 1$$
      \item [(iv)]
        \begin{question} $sg^*(x)$ \end{question}
        $$sg^*(x) = \abs{one(x) - sg(x)}$$
      \item [(v)]
        \begin{question} $C^n_k(x_1,\cdots,x_n)=k$ \end{question}
        $$C^n_k(x_1,\cdots,x_n) = s(C^n_{\abs{k-1}}(x_1,\cdots,x_n))$$
        $$C^n_0(x_1,\cdots,x_n) = 0$$
      \item [(vi)]
        \begin{question} $pred(x)$ \end{question}
        $$pred(x) = sg(x)\abs{x - 1}$$
        
    \end{enumerate}


  \item
    \begin{question}
      $R(x_1, \cdots, x_n)$ is elementary iff its characteristic function is elementary. Let $R_1(x_1, \cdots, x_n)$ and $R_2(x_1, \cdots, x_n)$ be elementary.
    \end{question}

    \begin{enumerate}
      \item
        \begin{question}
          Construct the characteristic functions for $\neg R_1(x_1,\cdots,x_n)$ and $R_1(x_1,\cdots,x_n) \land R_2(x_1,\cdots,x_n)$.
        \end{question}

        $$C_{\neg R_1}(x_1, \cdots, x_n) = \abs{one(x) - C_{R_1}(x_1, \cdots, x_n)}$$
        $$C_{R_1 \land R_2} = C_{R_1}(x_1, \cdots, x_n) C_{R_2}(x_1, \cdots, x_n)$$
      \item
        \begin{question}
          Show that if $R(x)$ is an arbitrary numerical relation and $\{ x : R(x) \}$ is finite, then $R(x)$ is elementary.
        \end{question}

        $R(x)$ is elementary iff $C_R(x)$ is elementary. First let's construct a relation: $E_k(x)$ holds iff $x=k$. We define its characteristic function as:
        $$C_{E_k}(x) = sg^*(\abs{x-k})$$
        Since its characteristic function is elementary, $E_k$ is elementary.\\
        Since there are a finite number of $x$ such that $R(x)$ holds, let's say, without loss of generalization, that they are: $x_1, \cdots, x_n$. First, we define:
        $$R_1 \lor R_2 \iff \neg(\neg R_1 \land \neg R_2)$$
        Finally, we can define $R(x)$ as follows:
        $$R(x) \iff E_{x_1}(x) \lor E_{x_2}(x) \lor \cdots \lor E_{x_n}(x)$$
        Since both $E_k$ and logical or are elementary, $R(x)$ must be elementary as well.
        
    \end{enumerate}
  \item 
    \begin{question}
      Show that the function $J(a,b)$ given by $\frac{1}{2}(a+b)(a+b+1) + a$ is onto.
    \end{question}

    First, we can show that $J$ is one-to-one. Say we have some $(a,b)$ and $(c,d)$ such that $J(a,b) = J(c,d)$. We can partition the set of all $(a,b,c,d)$ into 3 cases: either $(a+b) < (c+d)$, or $(a+b) = (c,d)$ or $(a+b) > (c+d)$.\\

    Let's examine the first case. Suppose that $(c+d)$ is $x$ larger than $(a+b)$. We can show that it's impossible that $J(a,b) = J(c,d)$ if $x > 0$:
    $$\frac{1}{2}(a+b+x)(a+b+x+1) - \frac{1}{2}(a+b)(a+b+1)$$
    $$= \frac{1}{2}x(2a + 2b + x + 1)$$
    If $x > 0$, then there is no way that $a$ can be large enough to make $J(a,b) = J(c,d)$ because the difference of the first monomials includes $a$ itself (from the $\frac{1}{2}(2a + \cdots)$). Because we've hit a contradiction, there's no way that $(a+b) < (c+d)$. And through symmetry, this also means that there's no way that $(c+d) < (a+b)$. So we now know that:
    $$J(a,b) = J(c,d) \implies (a+b) = (c+d)$$
    And if the sums are equal, then the first monomials must be equal ($\frac{1}{2}(a+b)(a+b+1) = \frac{1}{2}(c+d)(c+d+1)$), which then means that $a = c$. And if $a = c$ and $a+b = c+d$, then $b = d$.\\
    This means that $J(a,b) = J(c,d) \implies a=b \land c=d$, which is the definition of one-to-one functions.




    Let's consider all the terms of the sequence $(\frac{1}{2}(n)(n+1))$. The first is $\frac{1}{2}(0)(1) = 0$. And the distance between subsequent terms is:
    $$\frac{1}{2}(n+1)(n+2) - \frac{1}{2}(n)(n+1) = n + 1$$
    So inbetween subsequent terms, we can fit $n+1$ varying values of $a$ so that no values inbetween are left out. $a = 0, \cdots, n$, and we can define $b = n-a$.
\end{enumerate}

\end{document}