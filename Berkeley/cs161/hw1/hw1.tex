\input{../cs161.tex}
\usepackage{amsmath, dsfont, mathtools, verbatim, tikz, float}

\usetikzlibrary{arrows,automata}

\oddsidemargin 0in
\evensidemargin 0in
\textwidth 6.5in
\topmargin -0.5in
\textheight 9.0in
\newcommand{\norm}[1]{\left\lVert #1 \right\rVert}
\newcommand{\?}{\stackrel{?}{=}}
\DeclarePairedDelimiter{\ceil}{\lceil}{\rceil}

\begin{document}

\solution{Nikhil Unni}{Homework \#1}{Spring 2016}
\pagestyle{myheadings}

\begin{enumerate}
  \item
    Past-APF just meant that there's a difference between being able to access a message and understanding its meaning. Something that is secure should not be accessed by outside parties, but something that is private should not be able to be interpreted by third parties. What this implies about the design of cryptographic systems is that sometimes we cannot guard against parties snooping, but we can make sure that nothing cofidential is leaked. \newpage
  \item
    \begin{enumerate}
    \item True
    \item True
    \item False
    \item False
    \item False \newpage
    \end{enumerate}

  \item 
    \begin{enumerate}
      \item Eve, the passive eavesdropper cannot do anything to find the encrypted messages, since she cannot reverse the one-way functions. Mallory, on the other hand, can intercept the initial email exchange of public keys, and just send them her own public key. From then on, all subsequent communication is locked with Mallory's public key, and she has the private key to decrypt them.

      \item It is the same as part a -- Eve cannot do anything, but Mallory can. Just in this case, Mallory hijacks $g^x$ and $g^y$ and can send her own $g^m$ to both parties. \newpage
    \end{enumerate}


  \item 
    \begin{enumerate}
      \item
        Assuming that the plaintext is parsed from left to right, then we can have a username of ``al'', and a display name of ``Alice!A''. This results in a combined 10 bytes (with an exclamation mark separator), and the ``!0'' and random 4 bytes of padding will all be interpreted as garbage, and Alice will have administrator access.

      \item Since we have $C_0$, and $M_0$ (where $M_0$ is just the first 16 bytes of the original plaintext : ``alice!Alice X. J''), we can calculate AES$_K(IV + 0) = C_0 \oplus M_0$. Then, we just construct a new $M'_0 = \text{``alice!Alice!A000''}$, and subsequently, $C'_0 = \text{AES}_K(IV + 0) \oplus M'_0$. The following 16 bytes are just the original 16 bytes, since it doesn't depend on the previous 16 bytes. Therefore, we have:
        \begin{center}
          8452c68b712d36a0e0b900ff36dec332\\
          6df3d27be07554b6d25ce97b33ab70\\
          19892cfb1a95ac0e3ae97e3fb3bd9b88\\
        \end{center}
        The Python program to calculate it:
\begin{verbatim}
first = `alice!Alice X. J'
enc = `6df3d27be07554b6d25c0e96aa241b0a'
for i in range(0, len(enc), 2):
    split_up.append((enc[i] + "" + enc[i+1]).decode("hex"))
aes0 = [ord(first[i]) ^ ord(split_up[i]) for i in range(len(first))]
new_first = `alice!Alice!A000'
c0 = [hex((ord(new_first[i]) ^ aes0[i])) for i in range(16)]
print `'.join(a[2:] for a in c0)      
\end{verbatim}

      \item Say $IV$ and $M_0$ are the original initialization vector and message respectively. When decrypting $C_0$, we calculate $\text{DEC}(C_0) = IV \oplus M_0$. So if we hijacked the IV so that we can encrypt our new message, $M_0' = $ ``alice!A. Jones!A'', and construct $IV' = IV \oplus M_0 \oplus M_0'$. Then, when the server receives the original $C_0$ with our constructed $IV'$, we get, $M_{\text{received}} = \text{DEC}(C_0) \oplus IV' = (IV \oplus M_0) \oplus (IV \oplus M_0 \oplus M_0') = M_0'$.
        Therefore, we have: 
        \begin{center}
          8d681dc517d8b37bc21b1960a9757031\\
          3aa64478061df9515a55a347d55cbdee\\
        \end{center}
        
The Python program to calculate it:
\begin{verbatim}
first = `al1c3!A. Jones!O'
iv = `8d6845c541d8b37bc21b1960a975703f'
hex_iv = []
for i in range(0, len(iv), 2):
    hex_iv.append( int(iv[i]+iv[i+1] , 16) )
newPlain = `alice!A. Jones!A'
new_iv = [hex_iv[i] ^ ord(first[i]) ^ ord(newPlain[i]) for i in range(16)]
print ''.join(hex(a)[2:] for a in new_iv)
\end{verbatim}
    \end{enumerate}
\newpage

  \item
    \begin{enumerate}
      \item She just needs to calculate $B_1^{-k} * B_2$:
        $$ M = B_1^{-k} * B_2 = g^{-rk} * M * g^rk = M $$
      \item Say we have:
        $$C_1 = (C_{1,1}, C_{1,2}) = (g^r, M_1 * g^{rk})$$
        $$C_2 = (C_{2,1}, C_{2,2}) = (g^r, M_2 * g^{rk})$$
        Then Eve can calculate $g^{rk} = M_1^{-1} * C_{1,2}$, by calculating the multiplicative inverse of $M_1$, since this is known. From that, she can find $M_2$ like:
        $$M_2 = C_{2,2} * (g^{rk})^{-1}$$

      \item Say Mallory receives $C^* = (C_1^*, C_2^*) = (g^a, M * g^{ab})$ from Alice. Then, Mallory constructs a private key $x$, and public key $g^x$, and sends $(g^{ax}, M^x * g^{abx})$ it to Bob. Assuming that this is not a valid English sentence (and if it's not, Mallory can keep constructing new $x$ values until it's so), Bob will send back to Mallory the unencrypted message : $g^{-abx} * M^x * g^{abx} = M^x = M^x$. Since Mallory knows $x$, she can compute $M = (M^x)^{-x}$.
    \end{enumerate}

  \end{enumerate}
\end{document}
