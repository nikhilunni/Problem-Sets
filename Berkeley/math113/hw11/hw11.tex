\input{../math113.tex}
\usepackage{amsmath, amssymb, dsfont}

\newenvironment{amatrix}[1]{%
  \left(\begin{array}{@{}*{#1}{c}|c@{}}
}{%
  \end{array}\right)
}

\makeatletter
\renewcommand*\env@matrix[1][*\c@MaxMatrixCols c]{%
  \hskip -\arraycolsep
  \let\@ifnextchar\new@ifnextchar
  \array{#1}}
\makeatother

\newcommand{\?}{\stackrel{?}{=}}

\begin{document}

\solution{Nikhil Unni}{Homework \#10}{Fall 2015}
\pagestyle{myheadings}

\begin{enumerate}
  \item For each of the following use the modulo technique to show that the equation has no integer solutions.
    \begin{enumerate}
    \item $21x^2 - 36y = 44$\\\\
    
    We can show this has no integer solutions in $\mathds{Z}_3$. The equation becomes:
        $$21x^2 - 36y \equiv 44 (mod 3)$$
        $$0 - 0 \equiv 2$$
        $$0 \equiv 2$$
    ...which cannot be, regardless of our choice of x or y.\\
    
    \item $3x^2 - 4y = 5$\\\\
    
    If we take solve this in mod 4 (admittedly not a prime modulo group) then we get:
    
    $$3x^2 - 4y \equiv 5 (mod 4)$$
    $$3x^2 \equiv 1$$
    
    And looking at all the squares in mod 4:
    
    $$0^2 = 0$$
    $$1^2 = 1$$
    $$2^2 = 0$$
    $$3^2 = 1$$
    
    So $3x^2$ in mod 4 can only be $0$ or $3$, which is not 1. So there are no integer solutions in mod 4.
    
    
    \item $x^5 - 3y^5 = 2008$\\\\
    
    If we solve this in mod 11 we'll get:
    $$x^5 - 3y^5 \equiv 2008 (mod 11)$$
    $$x^5 - 3y^5 \equiv 6 (mod 11)$$
    
    Looking at all the powers of 5 in mod 11:
    
    $$0^5 = 0$$
    $$1^5 = 1$$
    $$2^5 = 10$$
    $$3^5 = 1$$
    $$4^5 = 1$$
    $$5^5 = 1$$
    $$6^5 = 10$$
    $$7^5 = 10$$
    $$8^5 = 10$$
    $$9^5 = 1$$
    $$10^5 = 10$$
    
    And looking at all the multiples of 3 of the possible power 5 values:
    $$3*0 = 0$$
    $$3*1 = 3$$
    $$3*10 = 8$$
    
    Looking at all the solutions of $\{0,1,10\} - \{0,3,8\}$, there's no way to get 6 in mod 11.\\
    Therefore, there can be no integer solutions in mod 11.
    
    \end{enumerate}
    
    
    \item Let $R = \mathds{Q}[x]$.
        \begin{enumerate}
        \item Prove that the set I of all polynomials in R which have 2 as a zero forms an ideal of R. Which of our adjectives for ideals (maximal, prime, principal) apply to I? Justify your answer.\\\\

          First we show that I is an additive subgroup. We know that $f(x) = 0$ is in I, because all $q \in \mathds{Q}$ are zeros of the 0 function. Then, for any $a,b \in I$:
          $$a-b = (x-2)f_a(x) - (x-2)f_b(x) = (x-2)((f_a-f_b)(x))$$
          We know that since 2 is a root of $a$ and $b$, we can factor out $x-2$. So the subtraction ends up being a multiple of $x-2$ as well, meaning that 2 is a root of $a-b$, making it a valid element of I.\\

          Next, we show it's an ideal. For $i \in I, r \in R$, $ir = ri$, since multiplication is commutative. Then:
          $$ir = ((x-2)f_i(x))*f_r(x)$$
          $$ir = (x-2)(f_i(x)f_r(x))$$
          This product, again, has $(x-2)$ as a factor, and thus has 2 as a root, making all $ir \in I$ as well.\\

          From the next problem, we see that $R/I$ is isomorphic to $\mathds{Q}$, which is a field. Since $I$ is maximal iff $R/I$ is a field, and since $\mathds{Q}$ is a field, we know that $I$ is maximal. From the definition from the book, an element, $a$, is a zero of $f(x)$ iff $(x-a)$ is a factor of $f(x)$. Definition chasing, this means that all elements $(x-2)f(x) \in R$ make up I.\\

          So, if we ever have an element $f(x)g(x) \in I$, if we split up $f(x)$ and $g(x)$ into its unique irreducible factors, $f(x)g(x)$ will be equivalent to $(x-2)f_1(x)g_1(x)f_2(x)g_2(x)\ldots$. So $(x-2)$ must have been a factor in either $f(x)$ or $g(x)$ (constant factors aside), meaning that if $fg(x) \in I$, either $f(x) \in I$ or $g(x) \in I$, making I prime (since $I \neq R$).\\

          By the same definition of an element of $I$, all elements in I are of the form $(x-2)f(x)$. So we can say that I is generated by $\langle (x-2) \rangle$, making I a principal ideal.\\
        
        \item What familiar ring is $R/I$ isomorphic to? Justify your answer.\\\\
          Let $\phi: R \rightarrow \mathds{Q}$, where $\phi(f(x)) \mapsto f(2)$, be the evaluation homomorphism at 2.\\
          For some $f(x), g(x) \in R$:
          $$\phi(f(x)g(x)) = f(2)g(2) = \phi(f(x))\phi(g(x))$$
          $$\phi(f(x) + g(x)) = f(2) + g(2) = \phi(f(x)) + \phi(g(x))$$
          $$\phi(1) = 1$$

          So we have a valid ring homomorphism.\\

          Then, we know that the kernel of $\phi$ is the set of all polynomials in R that map to 0 when evaluated at 2. And, by definition, these are the polynomials with 2 as a zero, or I. Also, we know that $\phi$ is onto, since we can just have $(x-a)$ as a term for all $a \in \mathds{Q}$, so by the First Isomorphism Theorem of Rings, we know that $\mathds{Q}$ is isomorphic to $R/I$.
        \end{enumerate}
        
    \item Let $R = \mathds{Q}[x,y]$, the ring of polynomials in two variables.\\\\

      Just as a note -- whenever I use $f(x)$ or $g(x)$ as examples of members of $R$, just know that $x \in \mathds{Q} \times \mathds{Q}$, and that it represents the x,y evaluation tuple, not just x... I just use it as shorthand for a generic polynomial in two variables.
        \begin{enumerate}
        \item Prove that $I = \{f(x,y) \in R : f(1,3) = f(2,-5) = 0 \}$ is an ideal of R.\\\\
          First we have to show it's an additive subgroup. We know that $0$ is in $I$, since it will evaluate to zero regardless of the arguments. Then, for some $a,b \in I$:
          $$(a-b)(1,3) = a(1,3) - b(1,3) = 0 - 0 = 0$$
          $$(a-b)(2,-5) = a(2,-5) - b(2,-5) = 0 - 0 = 0$$
          So it is a valid additive subgroup.\\
          
          Now we show it's a valid ideal in R. For $i \in I, r \in R$, $ir = ri$, since multiplication is commutative. Then:
          $$(ir)(1,3) = i(1,3)*r(1,3) = 0*r(1,3) = 0$$
          $$(ir)(2,-5) = i(2,-5)*r(2,-5) = 0*r(2,-5) = 0$$
          
          So I is a valid ideal of R.\\
          
        \item Is I equal to the principal ideal $\langle (3x-y)(5x+2y) \rangle$ in R? Justify your answer.\\\\
          It is not equal to that principal ideal, because we can find an element in I that is not the result of $(3x-y)(5x+2y)*f(x) = (15x^2 +xy -2y^2)f(x)$, for some $f(x) \in R$. If we choose our element to be $i = (3x^2-y)(5x+2y) = 15x^3 + 6x^2y - 5xy - 2y^2 \in I$, there's no way to find a multiple of the generator to get our element. It's a bit involved with a multivariate division algorithm, but if we just look at the monomials, we know we'll need an $x$ term with no coefficient to get $15x^3$ and a constant $1$ to get $-2y^2$, but these create a conflict when trying to find terms that might generate $6x^2y$ and $-5xy$.\\

        \item Can you determine whether I is maximal and/or prime?\\\\

          I is not maximal, since we can find a larger superset ideal. Let $J = \{f(x,y) \in R : f(1,3) = 0 \}$. This is a valid ideal for the same reason that we just showed in part (a) (subtracting out all of the $(2,-5)$ evaluations). It is clearly a superset of $I$, since it contains all the evaluations where $(1,3) \mapsto 0$, and $(2,-5) \mapsto 0$, but it also has elements where $(2,-5) \mapsto c$, for some $c \in \mathds{Q}$. So I cannot be maximal.\\

          Also, it is not prime. If we have some $f(x), g(x) \in R$ where $(fg)(x) \in I$, then $(fg)(1,3) = 0$ and $(fg)(2,-5) = 0$. This means:
          $$f(1,3)*g(1,3) = 0$$ 
          $$f(2,-5)*g(2,-5) = 0$$

          But if we had $f(x,y) = 3x-y$, and $g(x,y) = 5x+2y$, then $(fg)(x) \in I$, but $f(x) \not\in I$ and $g(x) \not\in I$.
          
        \end{enumerate}
\end{enumerate}

\end{document}
