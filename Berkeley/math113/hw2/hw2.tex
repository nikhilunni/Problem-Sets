\input{../math113.tex}
\usepackage{amsmath, dsfont}

\oddsidemargin 0in
\evensidemargin 0in
\textwidth 6.5in
\topmargin -0.5in
\textheight 9.0in
\newcommand{\norm}[1]{\left\lVert #1 \right\rVert}
\newcommand{\?}{\stackrel{?}{=}}

\begin{document}

\solution{Nikhil Unni}{Homework \#2}{Fall 2015}
\pagestyle{myheadings}


\begin{enumerate}
\item Let V denote the set of vectors $\langle x,y,z\rangle$ in $\mathds{R}^3$
, with the operations of addition and subtraction.
(Though vectors also have scalar multiplication, you can ignore that for this problem.)
Consider the relation\\
$\langle x_1, y_1, z_i\rangle \sim \langle x_2,y_2,z_2 \rangle$ if and only if $x_1-x_2 = y_1 - y_2 + k = z_1 - z_2 + 2k$ for some $k \in \mathds{R}$.

\begin{enumerate}
\item Show that $\sim$ is an equivalence relation on V.\\

To show that $\sim$ is an equivalence relation, we have to show that the operation is reflexive, symmetric, and transitive.\\\\
\textbf{Reflexive}:\\
For any $\langle x,y,z \rangle \in \mathds{R}^3$,\\
$$x-x \? y-y+k \? z-z+2k$$
$$0 \? k \? 2k$$
Which is always true if we select $k=0$.\\

\textbf{Symmetric}:\\
For any $\langle x_1,y_1,z_1 \rangle, \langle x_2,y_2,z_2 \rangle \in \mathds{R}^3$,\\
If $x_1-x_2=y_1-y_2+k_1=z_1-z_2+2k_1$, then we have to show that $x_2 - x_1 \? y_2 - y_1 + k_2 \? z_2 - z_1 + 2k_2$, which is the result of commuting our two vectors.
$$x_2 - x_1 \? y_2 - y_1 + k_2 \? z_2 - z_1 + 2k_2$$
$$= -x_2 + x_1 \? -y_2 + y_1 - k_2 \? -z_2 + z_1 - 2k_2$$
If we choose $k_2 = -k_1$, then we get back our original equation, which we know is true.
$$= -x_2 + x_1 = -y_2 + y_1 + k_1 = -z_2 + z_1 + 2k_2$$

\textbf{Transitive}:\\
For any $\langle x_1,y_1,z_1 \rangle, \langle x_2,y_2,z_2 \rangle, \langle x_3,y_3,z_3 \rangle \in \mathds{R}^3$,\\
If $$x_1-x_2=y_1-y_2+k_1=z_1-z_2+2k_1$$ and $$x_2-x_3=y_2-y_3+k_2=z_2-z_3+2k_2$$
then we have to show that
$$x_1-x_3 \? y_1-y_3+k_3 \? z_1-z_3+2k_3$$
(if $a \sim b, b \sim c$, then $a \sim c$).\\

If we simply add our first two equations we get:
$$x_1 - x_3 = y_1 - y_3 + k_1 + k_2 = z_1 - z_3 + 2k_1 + 2k_2$$
So if we simply let $k_3 = k_1 + k_2$, our unknown equation becomes true.

\item Describe the equivalence classes of $\sim$, giving a complete list. (Note: there are infinitely
many equivalence classes, so you will need set-building notation or something similar.)
Your explanation should make it clear that each vector of V belongs to exactly one
equivalence class from your list; or in other words, your alleged equivalence classes do
indeed partition V. Can you give both algebraic and geometric descriptions of the
equivalence classes?\\\\

Each of the equivalence classes are planes of the form $x - 2y + z = d$, for some $d \in \mathds{R}$. Because they all have the same normal to the plane, they are all parallel, and across the entire range of $d$ span the entire space of $\mathds{R}^3$. For this reason, points in $R^3$ are only part of a single class (plane), since all the planes are parallel, and the same point cannot fall on two parallel planes. Formally, the set of classes is:
$$\{\text{Planes of the form}\ x - 2y + z = d, d \in \mathds{R}\}$$
\end{enumerate}

\item
Use polar/exponential form for complex numbers in this problem ($r^{ei\theta}$ form).

\begin{enumerate}
\item Write out the binary operation tables for $Z_6 = \langle Z_6, +_6 \rangle$ and $U_6 = \langle U_6, \cdot \rangle$. Order the
rows and columns in a sensible way.\\

\begin{table}[h!]
  \begin{center}
    \begin{tabular}{|c|c|c|c|c|c|c|}
      \hline
      + & 0 & 1 & 2 & 3 & 4 & 5\\
      \hline  
      0 & 0 & 1 & 2 & 3 & 4 & 5\\
      1 & 1 & 2 & 3 & 4 & 5 & 0\\
      2 & 2 & 3 & 4 & 5 & 0 & 1\\
      3 & 3 & 4 & 5 & 0 & 1 & 2\\
      4 & 4 & 5 & 0 & 1 & 2 & 3\\
      5 & 5 & 0 & 1 & 2 & 3 & 4\\
      \hline
    \end{tabular}
    \caption{$\langle Z_6, +_6 \rangle$}
  \end{center}
\end{table}

\begin{table}[h!]
  \begin{center}
    \begin{tabular}{|c|c|c|c|c|c|c|}
      \hline
      $\cdot$       & 1              & $e^{2i(1\pi/6)}$ & $e^{2i(2\pi/6)}$ & $e^{2i(3\pi/6)}$ & $e^{2i(4\pi/6)}$ & $e^{2i(5\pi/6)}$\\
      \hline                  
      1             & 1 & $e^{2i(1\pi/6)}$ & $e^{2i(2\pi/6)}$ & $e^{2i(3\pi/6)}$ & $e^{2i(4\pi/6)}$ & $e^{2i(5\pi/6)}$\\
      $e^{2i(1\pi/6)}$ & $e^{2i(1\pi/6)}$ & $e^{2i(2\pi/6)}$ & $e^{2i(3\pi/6)}$ & $e^{2i(4\pi/6)}$ & $e^{2i(5\pi/6)}$ & 1\\
      $e^{2i(2\pi/6)}$ & $e^{2i(2\pi/6)}$ & $e^{2i(3\pi/6)}$ & $e^{2i(4\pi/6)}$ & $e^{2i(5\pi/6)}$ & 1 & $e^{2i(1\pi/6)}$\\
      $e^{2i(3\pi/6)}$ & $e^{2i(3\pi/6)}$ & $e^{2i(4\pi/6)}$ & $e^{2i(5\pi/6)}$ & 1 & $e^{2i(1\pi/6)}$ & $e^{2i(2\pi/6)}$\\
      $e^{2i(4\pi/6)}$ & $e^{2i(4\pi/6)}$ & $e^{2i(5\pi/6)}$ & 1 & $e^{2i(1\pi/6)}$ & $e^{2i(2\pi/6)}$ & $e^{2i(3\pi/6)}$\\
      $e^{2i(5\pi/6)}$ & $e^{2i(5\pi/6)}$ & 1 & $e^{2i(1\pi/6)}$ & $e^{2i(2\pi/6)}$ & $e^{2i(3\pi/6)}$ & $e^{2i(4\pi/6)}$\\
      \hline
    \end{tabular}
    \caption{$U_6 = \langle U_6, \cdot \rangle$}
  \end{center}
\end{table}

\item Prove that the map $f : \mathds{Z}_6 \rightarrow U_6$ given by $k \rightarrow (e^{\frac{2\pi}{6}i})^k$ is an isomorphism of binary structures.\\

Following the mapping, we can easily verify that the function f is one-to-one and onto:
$$\overline{0} \rightarrow 1$$
$$\overline{1} \rightarrow (e^{\frac{2\pi}{6}i})^1$$
$$\overline{2} \rightarrow (e^{\frac{2\pi}{6}i})^2$$
$$\overline{3} \rightarrow (e^{\frac{2\pi}{6}i})^3$$
$$\overline{4} \rightarrow (e^{\frac{2\pi}{6}i})^4$$
$$\overline{5} \rightarrow (e^{\frac{2\pi}{6}i})^5$$

To show isomorphism we have to verify $f(n+m) \? f(n) \cdot f(m)$.
$$(e^{\frac{2\pi}{6}i})^{n+m} \? (e^{\frac{2\pi}{6}i})^n \cdot (e^{\frac{2\pi}{6}i})^m$$
$$(e^{\frac{2\pi}{6}i})^{n+m} = (e^{\frac{2\pi}{6}i})^{n+m}$$
As you said in class, I won't show the wrapping around (as it's not strictly addition on $\mathds{Z}$), but it would be easy to show.\\

\item Find a different isomorphism of binary structures, $g : \mathds{Z}_6 \rightarrow U_6$, and prove your answer is an isomorphism.\\

Let our mapping, $\phi(n) = (e^{\frac{2\pi}{6}i})^{-n}$. Again, we can exhaustively show one-to-one and onto by listing out the mapping:
$$\overline{0} \rightarrow 1$$
$$\overline{1} \rightarrow (e^{\frac{2\pi}{6}i})^5$$
$$\overline{2} \rightarrow (e^{\frac{2\pi}{6}i})^4$$
$$\overline{3} \rightarrow (e^{\frac{2\pi}{6}i})^3$$
$$\overline{4} \rightarrow (e^{\frac{2\pi}{6}i})^2$$
$$\overline{5} \rightarrow (e^{\frac{2\pi}{6}i})^1$$

And we verify isomorphism with $f(n+m) \? f(n) \cdot f(m)$.
$$(e^{\frac{2\pi}{6}i})^{-(n+m)} \? (e^{\frac{2\pi}{6}i})^{-n} \cdot (e^{\frac{2\pi}{6}i})^{-m}$$
$$(e^{\frac{2\pi}{6}i})^{-n-m)} = (e^{\frac{2\pi}{6}i})^{-n-m}$$

\end{enumerate}

\item Let $\phi: \mathds{Z}_6 \rightarrow \mathds{Z}_6$ be defined by $\phi(n) = n - 3$ for all $n \in \mathds{Z}$. For each part below, construct a binary relation $*$ so that $\phi$ is an isomorphism of binary structures. Justify your answers.\\

\begin{enumerate}
\item $\phi: \langle \mathds{Z}, * \rangle \rightarrow \langle \mathds{Z}, + \rangle$\\

Let $n * m = n + m - 3$.
$$\phi(n*m) \? \phi(n) + \phi(m)$$
$$\phi(n + m - 3) \? n - 3 + m - 3$$
$$n + m - 6 = n + m - 6$$

\item $\phi: \langle \mathds{Z}, + \rangle \rightarrow \langle \mathds{Z}, * \rangle$\\

Let $n * m = n + m + 3$.
$$\phi(n+m) \? \phi(n) * \phi(m)$$
$$n + m - 3 \? (n-3) + (m-3) + 3$$
$$n + m - 3 = n + m - 3$$

\item $\phi: \langle \mathds{Z}, * \rangle \rightarrow \langle \mathds{Z}, \cdot \rangle$\\

Let $n * m = (n-3) \cdot (m-3)$.
$$\phi(n*m) \? \phi(n) \cdot \phi(m)$$
$$(n-3) \cdot (m-3) = (n-3) \cdot (m-3)$$

\item $\phi: \langle \mathds{Z}, \cdot \rangle \rightarrow \langle \mathds{Z}, * \rangle$\\

Let $n * m = (n+3) \cdot (m+3) - 3$.
$$\phi(n \cdot m) \? \phi(n) * \phi(m)$$
$$n \cdot m - 3 \? (n-3) * (m-3)$$
$$n \cdot m - 3 = n \cdot m - 3$$
\end{enumerate}

\end{enumerate}

\end{document}
