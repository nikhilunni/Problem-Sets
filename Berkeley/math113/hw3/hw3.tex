\input{../math113.tex}
\usepackage{amsmath, dsfont}

\newenvironment{amatrix}[1]{%
  \left(\begin{array}{@{}*{#1}{c}|c@{}}
}{%
  \end{array}\right)
}

\makeatletter
\renewcommand*\env@matrix[1][*\c@MaxMatrixCols c]{%
  \hskip -\arraycolsep
  \let\@ifnextchar\new@ifnextchar
  \array{#1}}
\makeatother

\begin{document}

\solution{Nikhil Unni}{Homework \#3}{Fall 2015}
\pagestyle{myheadings}


\begin{enumerate}
\item Let $G = \langle \mathds{R}^3, + \rangle$, the group of real 3-tuples with componentwise addition.
  \begin{enumerate}
  \item
    What geometric object is the set $$H = \{(x,y,z) \in \mathds{R}^3 : 3(x-1) + 2(y+5) - (z+1) = 6 \}$$
    Prove that H is a subgroup of G.\\\\

    The object is a plane within $\mathds{R}^3$.\\
    H is a subgroup of G if H is nonempty, and for all $a, b \in H$, $a+b^{-1} \in H$.
    For $a=(x_1,y_1,z_1), b=(x_2,y_2,z_2)$:
    $$3(x_1-1) + 2(y_1+5)-(z_1+1) = 6$$
    $$3(x_2-1) + 2(y_2+5)-(z_2+1) = 6$$
    $$a + b^{-1} = \langle x_1-x_2,y_1-y_2,z_1-z_2 \rangle$$
    Subtracting the second equation from the first we get:
    $$3(x_1-x_2) + 2(y_1-y_2) - (z_1-z_2) = 0$$
    $$3(x_1-x_2-1) + 2(y_1-y_2+5) - (z_1-z_2+1) = -3 + 10 - 1$$
    $$3(x_1-x_2-1) + 2(y_1-y_2+5) - (z_1-z_2+1) = 6$$

    Thus, $a + b^{-1} \in H$. 

  \item
    The cyclic subgroup generated by $(0,0,0)$ is the trivial subgroup of G, but other cyclic subgroups are more interesting. Let K be the cyclic subgroup generated by a nonidentity element $\vec{k}=(a,b,c)$. Give algebraic and geometric descriptions of K.\\\\

    All nontrivial K are a set of equally spaced points along the same line. Algebraically, they're all of the form:
    $$f(\vec{k}) = t \langle a,b,c \rangle, t \in \mathds{Z}$$
    
  \item
    Let L be the line through the origin and the point (2,3,5). is L a subgroup of G? If so, is it a cyclic subgroup of G?\\\\

    The particular equation for L is $t \langle 2,3,5 \rangle, t \in \mathds{R}$. We can show this is a subgroup similar to the proof in part a. Let $a=(2t_1,3t_1,5t_1),b=(2t_2,3t_2,5t_2) \in L$. Then:
    $$a + b^{-1} = \langle 2t_1-2t_2,3t_1-3t_2,5t_1-5t_2 \rangle$$
    $$a + b^{-1} = (t_1-t_2) \langle 2,3,5 \rangle$$
    Since $t_1-t_2 \in \mathds{R}$, $a + b^{-1} \in L$, and so it is a valid subgroup of G.\\
    However, L is not a valid cyclic subgroup of G. For that to be the case, there would have to be a single element that can be added to itself to produce every real number. This is only possible with countably infinite sets (or finite sets). Since the reals are uncountably infinite, no such generator can exist, meaning \textbf{L is not a valid cyclic subgroup of G}.

    
  \end{enumerate}

  \item
      Let $G = GL(3,\mathds{R})$, the group of 3x3 invertible matrices under multiplication.
    \begin{enumerate}
    \item
      Let L be the set of lower triangular 3x3 matrices with ones on the diagonal. Prove that L is a subgroup of G.\\

      First we have to show that L is a nonempty subset of G:
      $$
      \begin{pmatrix}[ccc|ccc]
        1 & 0 & 0 & 1 & 0 & 0\\
        a & 1 & 0 & 0 & 1 & 0\\
        b & c & 1 & 0 & 0 & 1\\
      \end{pmatrix}
      $$
      
      $$
      \begin{pmatrix}[ccc|ccc]
        1 & 0 & 0 &  1 & 0 & 0\\
        0 & 1 & 0 & -a & 1 & 0\\
        0 & c & 1 & -b & 0 & 1\\
      \end{pmatrix}
      $$

      $$
      \begin{pmatrix}[ccc|ccc]
        1 & 0 & 0 &  1 & 0 & 0\\
        0 & 1 & 0 & -a & 1 & 0\\
        0 & 0 & 1 & ac-b & -c & 1\\
      \end{pmatrix}
      $$

      Because all matrices in L are invertible 3x3 matrices (and substitute any real values for a,b,c to show it's nonempty), L is a subset of G.\\
      
      For $a,b\in L$ we must show that $ab^{-1} \in L$. Seeing as we just calculated the equation for the inverse of an element in L it's as simple as multiplying the two:
      $$
      \begin{pmatrix}[ccc]
        1 & 0 & 0 \\
        a & 1 & 0 \\
        b & c & 1 \\
      \end{pmatrix}      
      \begin{pmatrix}[ccc]
        1 & 0 & 0 \\
        x & 1 & 0 \\
        y & z & 1 \\
      \end{pmatrix}^{-1}
      $$
      
      $$
      \begin{pmatrix}[ccc]
        1 & 0 & 0 \\
        a & 1 & 0 \\
        b & c & 1 \\
      \end{pmatrix}      
      \begin{pmatrix}[ccc]
        1    &  0 & 0 \\
        -x   &  1 & 0 \\
        xz-y & -z & 1 \\
      \end{pmatrix}
      $$

      $$
      \begin{pmatrix}[ccc]
        1          & 0   & 0 \\
        a-x        & 1   & 0 \\
        b-cx+xz-y  & c-z & 1 \\
      \end{pmatrix}
      $$

      Because $\mathds{Z}$ is closed under addition/subtraction/multiplication, the result is a valid lower triangular matrix with ones on the diagonals, which means it's in L, which means L is a subgroup of G.\\


    \item
      Find two nontrivial abelian subgroups H and K of G which are not isomorphic to each other.
      $$
      H = 
      \left \{
      \begin{pmatrix}[ccc]
        1 & 0 & 0 \\
        0 & cos(\theta) & -sin(\theta) \\
        0 & sin(\theta) & cos(\theta) \\
      \end{pmatrix}
      \right \}
      $$
      Let H be the set of all rotation matrices around the x-axis. H is invertible, where the inverses are of the form:
      $$
      \begin{pmatrix}[ccc]
        1 & 0 & 0 \\
        0 & cos(\theta) & sin(\theta) \\
        0 & -sin(\theta) & cos(\theta) \\
      \end{pmatrix}
      $$

      Given $a,b \in H$, $ab^{-1} = $
      
      $$
      \begin{pmatrix}[ccc]
        1 & 0 & 0 \\
        0 & cos(\theta_1) & -sin(\theta_1) \\
        0 & sin(\theta_1) & cos(\theta_1) \\
      \end{pmatrix}
      \begin{pmatrix}[ccc]
        1 & 0 & 0 \\
        0 & cos(\theta_2) & sin(\theta_2)\\
        0 & -sin(\theta_2) & cos(\theta_2)\\
      \end{pmatrix}
      $$

      $$
      \begin{pmatrix}[ccc]
        1 & 0 & 0 \\
        0 & cos(\theta_1-\theta_2) & -sin(\theta_1-\theta_2) \\
        0 & sin(\theta_1-\theta_2) & cos(\theta_1-\theta_2) \\
      \end{pmatrix}
      $$
      Which is still in H, showing that \textbf{H is a subgroup of G.}

      $$
      K = 
      \left \{
      \begin{pmatrix}[ccc]
        1 & a & 0 \\
        0 & 1 & 0 \\
        0 & 0 & 1 \\
      \end{pmatrix}
      \right \}
      $$

      Let K be the set of all x-shear transformations. K is invertible, where the inverses are of the form:
      $$
      \begin{pmatrix}[ccc]
        1 & -m & 0 \\
        0 &  1 & 0 \\
        0 &  0 & 1 \\
      \end{pmatrix}
      $$

      Given $a,b \in K, ab^{-1} =$
      $$
      \begin{pmatrix}[ccc]
        1 & a_1 & 0 \\
        0 &  1 & 0 \\
        0 &  0 & 1 \\
      \end{pmatrix}
      \begin{pmatrix}[ccc]
        1 & -a_2 & 0 \\
        0 &  1 & 0 \\
        0 &  0 & 1 \\
      \end{pmatrix}
      $$

      $$
      \begin{pmatrix}[ccc]
        1 & a_1-a_2 & 0 \\
        0 &  1 & 0 \\
        0 &  0 & 1 \\
      \end{pmatrix}
      $$
      Which still satisfies the definition of a matrix in K, so \textbf{K is a subgroup of G}.\\

      We can show that H and K are not isomorphic. Let $r_{180}, r_0 \in H$ such that:
      $$
      r_{180} =
      \begin{pmatrix}[ccc]
        1 & 0 & 0 \\
        0 & cos(\pi) & -sin(\pi) \\
        0 & sin(\pi) & cos(\pi) \\
      \end{pmatrix}
      $$
      $$
      r_0 =
      \begin{pmatrix}[ccc]
        1 & 0 & 0 \\
        0 & cos(0) & -sin(0) \\
        0 & sin(0) & cos(0) \\
      \end{pmatrix}      
      $$

      Assume that there is a valid isomorphism between H and K. Then there must be a valid homomorphism, and at least one function $\phi$ that maps the elements from the set of rotation matrices to the set of shear-transformations such that $\phi(r_1r_2) = \phi(r_1)\phi(r_2)$.\\
      Then:
      $$\phi(r_{180}r_{180}) = \phi(r_{180})\phi(r_{180})$$
      $$\phi(r_0) = \phi(r_{180})\phi(r_{180})$$

      Because $r_0 = I$, and because the identity elements of subgroups to their matched groups are alligned, $\phi(I) = I$.

      Looking at the shear matrix form, the right-hand becomes:
      $$
      \begin{pmatrix}[ccc]
        1 & 2a & 0 \\
        0 &  1 & 0 \\
        0 &  0 & 1 \\
      \end{pmatrix}
      $$

      The only way for this to become the identity matrix is if $a = 0$, making $\phi(r_{180}) = I$. But this breaks the condition that the identity elements are mapped via $\phi$ in group isomorphism.\\

      Because there is an inconsistency, our assumption was incorrect (we've disproved the contrapositive), meaning there is no valid isomorphism between H and K.      
    \end{enumerate}

  \item Let the function $\phi: G \rightarrow H$, where G and H are groups, be a group homomorphism.

    First, it can be proved that the identity elements are mapped by $\phi$:
    $$\phi(e_ga) = \phi(e_g)\phi(a)$$
    $$\phi(a)\phi(a)^{-1} = \phi(e_g)\phi(a)\phi(a)^{-1}$$
    $$e_h = \phi(e_g)$$
    
    \begin{enumerate}
    \item
      Prove that if a and b are inverses of each other in G, then $\phi(a)$ and $\phi(b)$ are inverses of each other in H. Conclude that if an element x is its own inverse in G, then $\phi(x)$ is its own inverse in H.\\\\

      $$\phi(ab) = \phi(a)\phi(b)$$
      $$\phi(e_g) = \phi(a)\phi(b)$$
      $$e_h = \phi(a)\phi(b)$$

      So $\phi(a)$ and $\phi(b)$ are inverses. Substitute b with a above, and we've proven the conclusion.

    \item
      If G is abelian, prove that $\phi(g_2)\phi(g_1) = \phi(g_1)\phi(g_2)$. Does this prove that H is also abelian?\\\\

      $$\phi(g_2g_1) = \phi(g_1g_2)$$
      Because G commutes, $\phi$ has to map to the same consistent value.
      $$\phi(g_2g_1) = \phi(g_2)\phi(g_1)$$
      $$\phi(g_1g_2) = \phi(g_1)\phi(g_2)$$
      $$\phi(g_2)\phi(g_1) = \phi(g_1)\phi(g_2)$$

      Because $\phi$ is a valid isomorphism, the function is one-to-one and onto. So by proving the above, for all $\phi(g_1), \phi(g_2) \in H$, we've showed that it holds true for every pair of elements $h_1, h_2 \in H$, which means that H is abelian as well.
      
    \end{enumerate}

\end{enumerate}

\end{document}
